\section{\blurname Design}

\begin{comment}
\subsection{Ubiquitous Sensing}

{\bf Motivation. }Smart devices today provide users with pervasive
connectivity. 
An understanding of smartphones would give great insights into problems 
across disciplines.   For example, urban scientists can investigate real-time 
traffic patterns~\cite{zhuang2011time}, 
public safety agencies can benefit from low-cost, low-latency emergency 
message dissemination~\cite{zhuang2010probabilistic}, and broadcasting audio /
video 
clips of life-threatening hazard~\cite{zhuang2012uplink} in disaster response.
Additionally, sensor data has significant value in many other domains: 
it enables mobile application performance monitoring
\cite{ravindranathappinsight}, 
increases cost-effectiveness of network deployment for telecom and mobile
service providers \cite{Sieb05}, enhances location and social awareness within
applications from a social science standpoint \cite{hsu2007mining}, and enables
remote health monitoring and remote diagnosis for health professionals and
doctors \cite{infections, health}, etc.  
Due to the lack of a unified instrument that supports these  
applications, researchers and scientists cannot fully explore such potential 
that could tremendously improve the quality of people's life.  
The \sensorname project~\cite{seattle-sensor-project, sensor} will leverage
\sysname to allow uniform sensor data access 
between different research groups across
domains.

{\bf Overview of the Solution.}
\sensorname will allow a researcher who wants to
build a sensing application the ability to write the majority of their
application in \sensorname, yet be able to execute it across Android, iOS, 
Maemo, and other mobile operating systems.  
% 
Better yet, gaining end user adoption does not need
to be done on a per-project basis.   The Seattle Sensors project
builds a common framework~\cite{rafetseder2013sensorium} that allows researchers
to share their sensor data 
across diverse projects.   This will allow a single action by a user to 
responsibly share their sensor data with a broad array of
social, economic, computer, and urban scientists. 

{\bf Related Work. }
Research projects such as 
PhoneLab~\cite{phonelab} and Mobile Territorial Lab (MTL)~\cite{mtl} are just 
starting to deploy similar services on mobile platforms to collect 
sensor-related data from end users. However, they only 
provide partial solutions and lack a viable plan for broad external use --- 
the user
base of these platforms are only limited to committed users. Their deployment
model is
rather constrained as researchers %are limited to a few target apps, and they 
cannot participate proactively and deploy services tailored to different needs
for
their research and development. As a result, the capacity for researchers to 
leverage each others' deployment is lacking. 

{\bf What \sensorname Provides.}
The \sensorname instrument provides a simplified and standard interface to
access 
sensors on end user's smartphones that allow it.
The sandbox interacts with the physical sensors on the device through a 
common, simplified API, which provides access to sensors
in the same way on different devices (an XML-style interface).
Even though data types may be different depending on the physical sensors,
the researcher may get the data in a uniform manner.   For example,
on an Android phone in a campus area, this library may automatically
supplement GPS data with WiFi signal and location information~\cite{open3g,
rafetseder2013sensorium}
to provide more accurate information.   However, on a iPhone in a metropolitan
area, 4G and WiFi triangulation may prove more accurate.   Libraries can
automatically
improve, refine, and equivocate contextual sensor data from diverse devices.  
As a result
a researcher can write their sensing code once, and get highly accurate 
data across a diverse set of platforms.

{\bf What Participants Build.}  
A researcher's program can simply import the libraries for 
reading sensor data and trivially access it.   Since the client
software and sensor libraries make the data uniform, the exact same
program will work on Android, iPhone, iPad, Nokia, Win 8,
and other supported devices.  
A researcher can also trivially exfiltrate sensor data back
to a server of their choice (such as those provided by CUSP~\cite{cusp})  
using our XMLRPC libraries.   This is handy
for archival or IRB-approved data sharing.
By pooling sensors across a wide array of end users who opt-in (as being
developed by the Seattle Sensors project),
a researcher can deploy a
sensing application across thousands of smartphones around the world
in minutes.
\end{comment}

\subsection{\blurname Overview} 

\subsubsection{Overview of the Solution}

Today's smartphone OSes typically expose resources in a global way. 
For example, apps in Android~\cite{android-os} use 
install-time manifests to request access to resources; once granted,  
the installed app has permanent access to the requested resources.  
Such permissions are often much more than necessary. Several related work has 
been proposed to refine or reduce permissions on mobile
platforms~\cite{hornyack2011these, zhou2011taming, conti2011crepe,
pearce2012addroid} via modifying the device platform. BlurSense allows untrusted
parties to add privacy filters from user space.  Multiple security vendors can
efficiently and effectively collaborate to strengthen user privacy.

Different from the existing privacy controls, \blurname provides a generic
privacy protection framework called BlurSense via a richer set of control of
privacy by third parties. Users not only gain full transparency of what
information is captured on their devices, but also have full control over how
much information they would share with the rest of the world --- a secure
personal data ecosystem. After installing BlurSense, a user can install software
from a third party (like a security vendor) that performs custom sensor
filtering actions in response to application requests. For example, an
application could be prevented from using motion sensors when in the background,
or precise GPS data could be abstracted to a neighborhood or zipcode. BlurSense
provides effective controls for smartphones, much like Flash and JavaScript
filtering tools protect laptops and desktops (e.g., NoScript~\cite{noscript},
AdBlock~\cite{adblock}, FlashBlock~\cite{Flashblock}). 


\subsubsection{What \blurname Provides}

Seattle provides a sensor interposition mechanism and a sandboxing mechanism
that make it easy to implement privacy filters. A user can let a third party
have access to their sensor data from within a security and performance isolated
container~\cite{Cappos_CCS_10}.   By leveraging this security, the user provides
minimal trust in the third party, but allows them an easy way to code their
filters.   This functionality also automatically handles multiple privacy
filters from different parties through the sandbox policy composition
functionality~\cite{Cappos_CCS_10}.

% {\bf What Participants Build.} 
Participants will build a mechanism to trap the requests from generic
applications and pass them into BlurSense (a proof-of-concept for 
Android has already been built). 
Participants will build and manage an ``App Store'' for BlurSense to allow 
users to locate privacy filters they wish to apply.   These may range from 
sharing sensor data with researchers by reducing
the precision of sensor values, salting and hashing sensor values for 
anonymizing collected data, or completely denying access to individual 
(or all) sensors. For a particular sensor, a filter might perform an action such
as blurring the resolution of photos and video taken by the camera, 
removing access point information from WiFi scans, or
omitting the motion sensor data completely.   Security and privacy groups can
easily build and disseminate their own privacy filters they recommend to users
by adding them to BlurSense app store. Therefore, \blurname is able to handle 
the three categories of threats in Section~\ref{sec-threat}. 

\subsection{Design Framework}