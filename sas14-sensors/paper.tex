\documentclass[conference]{IEEEtran}
\usepackage{epsfig,endnotes}
\usepackage[hyphens]{url}
\usepackage{color}
\usepackage{comment}
\usepackage{float}
\usepackage{subfig}
\usepackage{amsmath}
\usepackage{lipsum}
\usepackage{pifont} %used for tickmark and crosses.
\usepackage{array}
\usepackage{multirow}
\usepackage{amssymb}
\usepackage{amsthm}
\usepackage{amsmath}
\usepackage[normalem]{ulem}
\usepackage{xspace}
\usepackage{fancyvrb}
\usepackage{fixltx2e}

\hyphenation{op-tical net-works semi-conduc-tor}
\newcommand{\sysname}{Sensibility Testbed\xspace}

\newcommand{\sensorname}{Sensorium\xspace}
\newcommand{\blurname}{BlurSense\xspace}

%%%%%%%%%%%%%%%%%%
\usepackage{hyperref}

\hypersetup{%
  pdftitle = {TBD},
  pdfkeywords = {},
  pdfauthor = {TBD},
  bookmarksnumbered,
  bookmarksopen=true,
  colorlinks=true,
  urlcolor=[rgb]{.35,0,0},
  linkcolor=[rgb]{.35,0,0},
  citecolor=[rgb]{.35,0,0},
  pdfstartview={FitH},
}

\newcommand{\eat}[1]{}
\newcommand{\rmv}[1]{ #1}

\newif\ifrev
% comment out the following line after the revision is done
\revtrue
\ifrev
  \newcommand{\cappos}[1]{{\color{red} [Justin: #1]}}
  \newcommand{\yanyan}[1]{{\color{blue} [Yanyan: #1]}}
\else
  \newcommand{\cappos}[1]{}
  \newcommand{\yanyan}[1]{}
\fi

\begin{document}

\title{\blurname: Dynamic Fine-Grained Access Control to Protect 
Smartphone Users’ Privacy}

\author{\IEEEauthorblockN{Homer}
\IEEEauthorblockA{Twentieth Century Fox\\
Springfield, USA\\
Email: homer@thesimpsons.com}
\and
\IEEEauthorblockN{Bart}
\IEEEauthorblockA{Twentieth Century Fox\\
Springfield, USA\\
Email: bart@thesimpsons.com}}

\maketitle


\begin{abstract}
For many people, smartphones serve as a technical interface to the modern world.
Access to devices at the network edge, such as tablets and smartphones, is
immensely valuable to researchers. More importantly, these smart devices also
have embedded on-board sensors, such as accelerometer, gyroscope, GPS, and
camera, which enable \yanyan{find a better word than ``enable''.} the
development of innovative mobile applications. 

This work builds \blurname, a tool that provides secure and customizable access
to a large number of sensors on smartphones, tablets, and similar end user
devices. The current access control to the smartphone resources, such as sensor
data, is static and coarse-grained. \blurname is a dynamic, fine-grained access
control mechanism, acting as a defensive line that allows to define and add
privacy filters from user space. As a results, any user can can securely expose
sensor data to the outside world, and any researchers can securely share the
privacy filters with other researchers to improve the capabilities of each
other's experiment.
\yanyan{need work on the last sentence...doesn't sound right.}
\end{abstract}

\IEEEpeerreviewmaketitle

\section{Introduction}
Smartphones and tablets are becoming the dominant way that people interact with
the physical world.  This trend is showing no sign of stopping.  Smartphones
overtook PC sales in 2011, and even tablets will outsell desktop PCs by
2013~\cite{phonesales}. As of October 2012, there were more than one billion
smartphones in use~\cite{smartphones-use}. By the end of 2013, the number of
mobile devices (like smartphones and tablets) is projected to surpass the number
of people on the planet~\cite{cisco-data}. Due to such ubiquity across
demographic and geographic spectrum, utilizing smartphones would substantially
benefit science and humanity.

\begin{comment}
First, smart devices produce immensely valuable data for researchers across
a wide variety of disciplines. As an interface between our everyday life and the
physical world, these devices present valuable data for 
critical scientific and social pursuits.
If properly harnessed, the embedded cellular, GPS, WiFi, Bluetooth, camera
sensors, etc.,
can objectively record information gathered from a user's
perspective for the benefit of all. For example, it is possible for researchers
to 
understand carrier and device performance 
differences~\cite{huang2010anatomizing}, increase wireless network performance
leveraging
user mobility~\cite{ravindranath2011improving}, improve navigation and
personalized 
trip management~\cite{thiagarajan2011accurate} and detect earthquakes and 
tsunamis accurately~\cite{faulkner2011next}. Estimation of traffic in an 
area can help transportation scientists refine usage models for environmental
impacts~\cite{lena2002elemental}. 
Health sciences, sociology, or any domain concerned with the movement of people
and the associated impact could be explored~\cite{goldman2009participatory}. 
But until now, there was no unified way to harvest these opportunities. 

Second, the computational power of modern smartphones is substantial.   
These devices provide significant
capacity and computing power at network edge --- 
a fact that has largely missed, except by criminals~\cite{botnet}.
Although other services like cloud computing provide substantial computing 
power, the location
of cloud infrastructure is less than ideal for many users because
of the latency when accessing many cloud datacenters.  Dedicated cloud 
infrastructure
also has limited host and network heterogeneity, limited scalability, and 
lack a deployment path to end-user systems. From the success of CDNs,
the trend has been toward pushing small datacenters close to end
users~\cite{edge-cloud}.
Smartphones, laptops, and tablets are already on the same network as
the user and thus minimize latency.
If researchers could use these devices to measure networks 
and serve content, this would provide deep insight into parts of the Internet 
that are currently invisible~\cite{bauermeasuring}. Therefore, harnessing the
power of smartphones 
provides unique opportunities for improving the quality of life
and enhancing the operation of modern society.

We propose to develop \sysname that will unleash the potential of
tablets and smartphones. 
Drawing on our experience (and user base) from our
Seattle platform, \sysname will control access to computational sandboxes
that runs unintrusively in the background of the tablets and smartphones of
users around the world.   
Any researcher can quickly and use the \sysname to easily access 
resources across a huge and diverse set of smartphones and tablets.   
(In fact thousands of researchers are already using our current 
testbed~\cite{SeattleClearinghouse}.)
This will provide unprecedented access to information about 
wireless access networks in diverse geographic regions, 
community roles and environmental impacts of individuals, 
and social patterns people  all around the world.
As a result, the \sysname will provide unparalleled research
capabilities for scientists and researchers across a wide variety of fields
\end{comment}

Smartphones and tablets in the modern age have powerful capabilities in
computing, communication, and sensing~\cite{cai2009defending}. Like desktop and
laptop machines, they can perform complex computing tasks according to user
commands. These smart devices can also communicate with each other through
wireless communications. Unlike regular computers, they have embedded
onboard sensors, such as accelerometer, gyroscope, GPS, and camera, which
enables the development of innovative mobile applications~\cite{sensor}.
Although the sensing capabilities enhance the convenience of user interfaces and
application usefulness, they also raise serious privacy
concerns~\cite{shabtai2010google}. For instance, through accessing sensor data,
malicious applications could retrieve sensitive information about the mobile
phone users, such as location, input passwords, and credit card numbers. They
even might be able to send these sensitive information to remote
attackers~\cite{xu2012taplogger, miluzzo2012tapprints, xu2009stealthy,
schlegel2011soundcomber, cai2011touchlogger, marquardt2011sp}. 

The current access control to the smartphone resources, such as sensor data, is
static and coarse-grained. We call this \textit{the first defensive line}. Take
the Android platform as an example, the access permissions are either granted or
denied completely during the installation of applications based on a request XML
manifest file. As a result, applications may ask for more permissions than
needed by the requirement. Once granted the requested permissions, applications
have access to those resources permanently. Another issue of the current work is
that they require modifications to the Android platform, such as
in~\cite{conti2011crepe, hornyack2011these}. This increases the cost of
maintenance and cannot be used in legacy systems. To protect smartphone
users' privacy, we propose a \textit{second defensive line} behind the first
line: a dynamic, fine-grained access control mechanism, which allows to
define and add privacy filters from user space~\cite{blur, Cappos_CCS_10}.


Specifically, we will implement a framework of reference monitors, to enforce
mandatory access control to sensor data in real time. The subjects of our
framework will be the processes of applications in user space. The objects will
be the sensor data on smart devices. If a process of any application needs to
access any of the sensor data, our second defensive line will come into play,
mediating every access to sensor data as follows. 

According to the semantics of the access requests and the current context,
different reference monitors will be triggered, calculating a sensitivity score
for each request from the applications, which has got through the first
defensive line. If the request is for highly sensitive sensor data, then the
request might be simply rejected. Otherwise, if the request is for medium-level
sensitive sensor data, the request might get through, but the returned data is
bogus. If the request is for low-sensitive sensor data, then the return results
need to be processed, e.g., by obfuscation, and the obfuscation ratio is
proportional to the sensitivity of the request: if the request is more
sensitive, then the returned data will be obfuscated in a more significant way.
If the request is for non-sensitive sensor data, then the unprocessed sensor
data will be returned.
\section{System Architecture and Threat Model}

In this section, we first briefly describe \sensorname, the basis of \blurname.
Then we discuss the new extensions to \sensorname for enabling \blurname through
defining a threat model.

\subsection{Building \sensorname}

The prerequisite of \blurname is to gain access to sensor data. However,
different devices and platforms, such as Android and iOS, use very different
interfaces into their sensors. One of the major goals of our testbed is to
support a wide range of sensor categories, broader device and network diversity,
while the client software still behave in a portable manner.  

\subsubsection{Seattle Porting onto Mobile Devices}\label{sec-portability}
Seattle is the testbed platform we have developed over the past four
years~\cite{Seattlewebpage}. It supports a wide range of devices including
desktop, laptop, servers, etc. We recently ported Seattle onto mobile platforms.

Compared to desktop and laptop environments, development on mobile platforms has
more difficulties and restrictions due to the inherent resource limitations,
such as limited computational power and battery levels. 
However, researchers were able to do early stage ports 
of Seattle to Android~\cite{seattleonandroid} (and jailbroken iPad / iPhone / 
iPod) with a few weeks of developer effort. Users can now download a 
native Android installer (APK) from the Google Play 
store~\cite{seattle-android}. Our Seattle testbed on Android supports Android 
versions from 2.1 to 4.0.4 (API levels 7 to 15), covering device versions with 
the highest market distribution~\cite{dashboard-android}.   
Despite never being advertised or mentioned publicly, our Seattle 
app in Google Play has more than 50 installs.   
% While not yet a production-ready release, the proof-of-concept implementations
% on Android, iPad, iPhone, and OpenWRT home wireless routers, demonstrate
% this is a feasible, low-risk effort.

\subsubsection{Supporting Sensors on Mobile Devices}

Due to the isolation of the VM in Seattle, a researcher cannot normally access
sensors on a user's mobile device, such as GPS, WiFi SSID and signal strength,
motion sensors (e.g., accelerometer, compass, gyroscope, barometer), etc.
However, many users would like a way to provide information about these sensors
to selected researchers. To facilitate the exchange of information, we provide a
set of API that reads sensor data from a Seattle VM for researchers trusted by
the user. This API framework is \sensorname. 

\sensorname is a generic sensor reading framework built on top of Seattle on
Android. It funnels data from actual sensor drivers, implements fine-grained
privacy control for the user, and provides generic outbound interfaces such as
XML-RPC. However, the APIs provided by native smartphone sensors vary
significantly across platforms. Our philosophy is to provide a simple API that  
would allow a variety of sensor applications to operate in a unified manner.

First, we implemented system hooks called {\it sensor modules} to interact with
a variety of sensors through system programming interfaces. Currently,
implemented sensor modules and the available contextual information are
classified into three categories: device specific (percentage of battery power
level, CPU and memory utility), location related (latitude, longitude, altitude,
accuracy, and speed if available), and network related (mobile network type and
operator, nearby WiFi access point and Bluetooth devices). While sensor modules
are the system hooks with read access to valuable sensor resources, they cannot
manipulate sensor data. Additionally, the sensor API also provides a base {\it
registry service} with a common interface which each sensor implementation can
easily be plugged into.  For both local and remote processes to access sensor
data, an XML-RPC library~\cite{xmlrpc-android} is incorporated to provide data
in an unified format. In case newer sensors appear on future mobile devices,
developers can add newly implemented sensors into this framework rather easily.
The registry service listens for connection on a set of predefined ports via
XML-RPC. Thus, both local and remote process can connect to these ports and
register for sensor updates. 

Our preliminary work in this area has resulted in working 
code~\cite{seattle-sensor-git}, tutorials~\cite{seattle-sensor-project}, and a 
blog for problem discussion~\cite{sensor}. Several different groups have
already used our early-stage proof-of-concept to solve problems across a variety
of domains, demonstrating the potential of sharing sensor data. 

\subsection{\blurname Threat Model}\label{sec-threat}

While sensors on smartphones are a powerful tool for researchers, they also pose
a risk to users. A study showed that unscrupulous hackers typically find
personal information stored on devices inviting~\cite{mulligan2000your}. There
has been alarming news about privacy breaches of personal data on smart devices:
26\% of Android apps in Google Play can access user's personal
data~\cite{toomuch}; an iOS app auto-posts false piracy accusations on users'
Twitter accounts~\cite{tweetios}; apps can steal sensitive information like
passwords using the smartphone's motion sensors to determine tapped
keys~\cite{xu2012taplogger}; and a huge botnet that is collecting sensor data
was discovered on more than a million end user smartphones~\cite{botnet}. The
Federal Trade Commission (FTC) recently recommended that mobile platforms should
provide in-time disclosures to users of accessing sensitive content on smart
devices~\cite{ftc}. However, the current privacy mechanisms are rather limited,
and most privacy controls are one-size-fits-all: the user either opts-in or
cannot use the site or application. Therefore, care must be taken when sharing
data on smartphones and similar devices. And there is a need to control the
access of third party in a consistent but also dynamic, fine-grained manner.
\yanyan{perhaps shorten this.}

To further motivate our work, we consider three cases in our threat model. The
first category under consideration is called \textbf{ambitious legitimate
applications}. In this category, the developers of the applications ask for more
permissions than needed in the manifest file, e.g., for profit purpose. For
example, the applications run some hidden code, connecting to remote
advertisement servers, which analyze smartphone users' characters based on their
behavioral patterns. The servers will then send these users advertisements
that they might be interested in, to increase the income of those companies who
posted the advertisements.

The second category is called \textbf{compromised legitimate applications}. In
this category, the legitimate applications have some vulnerabilities, such as
buffer overflow, so these legitimate applications are actually under the full
control of remote attackers. The attackers might bypass the first defensive line
and collect sensitive information about the users, such as contacts, credit card
numbers, and passwords, to jeopardize users’ privacy.

The third category is called \textbf{malicious applications}. These applications
are designed by attackers with malicious purposes. They behave like benign ones,
but in fact, they covertly collect users’ private information, like trojans.

In the following section, we introduce our proposed scheme that is able to
detect and defend against all these three types of threats above.
\section{\blurname Design}

\begin{comment}
\subsection{Ubiquitous Sensing}

{\bf Motivation. }Smart devices today provide users with pervasive
connectivity. 
An understanding of smartphones would give great insights into problems 
across disciplines.   For example, urban scientists can investigate real-time 
traffic patterns~\cite{zhuang2011time}, 
public safety agencies can benefit from low-cost, low-latency emergency 
message dissemination~\cite{zhuang2010probabilistic}, and broadcasting audio /
video 
clips of life-threatening hazard~\cite{zhuang2012uplink} in disaster response.
Additionally, sensor data has significant value in many other domains: 
it enables mobile application performance monitoring
\cite{ravindranathappinsight}, 
increases cost-effectiveness of network deployment for telecom and mobile
service providers \cite{Sieb05}, enhances location and social awareness within
applications from a social science standpoint \cite{hsu2007mining}, and enables
remote health monitoring and remote diagnosis for health professionals and
doctors \cite{infections, health}, etc.  
Due to the lack of a unified instrument that supports these  
applications, researchers and scientists cannot fully explore such potential 
that could tremendously improve the quality of people's life.  
The \sensorname project~\cite{seattle-sensor-project, sensor} will leverage
\sysname to allow uniform sensor data access 
between different research groups across
domains.

{\bf Overview of the Solution.}
\sensorname will allow a researcher who wants to
build a sensing application the ability to write the majority of their
application in \sensorname, yet be able to execute it across Android, iOS, 
Maemo, and other mobile operating systems.  
% 
Better yet, gaining end user adoption does not need
to be done on a per-project basis.   The Seattle Sensors project
builds a common framework~\cite{rafetseder2013sensorium} that allows researchers
to share their sensor data 
across diverse projects.   This will allow a single action by a user to 
responsibly share their sensor data with a broad array of
social, economic, computer, and urban scientists. 

{\bf Related Work. }
Research projects such as 
PhoneLab~\cite{phonelab} and Mobile Territorial Lab (MTL)~\cite{mtl} are just 
starting to deploy similar services on mobile platforms to collect 
sensor-related data from end users. However, they only 
provide partial solutions and lack a viable plan for broad external use --- 
the user
base of these platforms are only limited to committed users. Their deployment
model is
rather constrained as researchers %are limited to a few target apps, and they 
cannot participate proactively and deploy services tailored to different needs
for
their research and development. As a result, the capacity for researchers to 
leverage each others' deployment is lacking. 

{\bf What \sensorname Provides.}
The \sensorname instrument provides a simplified and standard interface to
access 
sensors on end user's smartphones that allow it.
The sandbox interacts with the physical sensors on the device through a 
common, simplified API, which provides access to sensors
in the same way on different devices (an XML-style interface).
Even though data types may be different depending on the physical sensors,
the researcher may get the data in a uniform manner.   For example,
on an Android phone in a campus area, this library may automatically
supplement GPS data with WiFi signal and location information~\cite{open3g,
rafetseder2013sensorium}
to provide more accurate information.   However, on a iPhone in a metropolitan
area, 4G and WiFi triangulation may prove more accurate.   Libraries can
automatically
improve, refine, and equivocate contextual sensor data from diverse devices.  
As a result
a researcher can write their sensing code once, and get highly accurate 
data across a diverse set of platforms.

{\bf What Participants Build.}  
A researcher's program can simply import the libraries for 
reading sensor data and trivially access it.   Since the client
software and sensor libraries make the data uniform, the exact same
program will work on Android, iPhone, iPad, Nokia, Win 8,
and other supported devices.  
A researcher can also trivially exfiltrate sensor data back
to a server of their choice (such as those provided by CUSP~\cite{cusp})  
using our XMLRPC libraries.   This is handy
for archival or IRB-approved data sharing.
By pooling sensors across a wide array of end users who opt-in (as being
developed by the Seattle Sensors project),
a researcher can deploy a
sensing application across thousands of smartphones around the world
in minutes.
\end{comment}

\subsection{\blurname Overview} 

\subsubsection{Overview of the Solution}

Today's smartphone OSes typically expose resources in a global way. 
For example, apps in Android~\cite{android-os} use 
install-time manifests to request access to resources; once granted,  
the installed app has permanent access to the requested resources.  
Such permissions are often much more than necessary. Several related work has 
been proposed to refine or reduce permissions on mobile
platforms~\cite{hornyack2011these, zhou2011taming, conti2011crepe,
pearce2012addroid} via modifying the device platform. BlurSense allows untrusted
parties to add privacy filters from user space.  Multiple security vendors can
efficiently and effectively collaborate to strengthen user privacy.

Different from the existing privacy controls, \blurname provides a generic
privacy protection framework called BlurSense via a richer set of control of
privacy by third parties. Users not only gain full transparency of what
information is captured on their devices, but also have full control over how
much information they would share with the rest of the world --- a secure
personal data ecosystem. After installing BlurSense, a user can install software
from a third party (like a security vendor) that performs custom sensor
filtering actions in response to application requests. For example, an
application could be prevented from using motion sensors when in the background,
or precise GPS data could be abstracted to a neighborhood or zipcode. BlurSense
provides effective controls for smartphones, much like Flash and JavaScript
filtering tools protect laptops and desktops (e.g., NoScript~\cite{noscript},
AdBlock~\cite{adblock}, FlashBlock~\cite{Flashblock}). 


\subsubsection{What \blurname Provides}

Seattle provides a sensor interposition mechanism and a sandboxing mechanism
that make it easy to implement privacy filters. A user can let a third party
have access to their sensor data from within a security and performance isolated
container~\cite{Cappos_CCS_10}.   By leveraging this security, the user provides
minimal trust in the third party, but allows them an easy way to code their
filters.   This functionality also automatically handles multiple privacy
filters from different parties through the sandbox policy composition
functionality~\cite{Cappos_CCS_10}.

% {\bf What Participants Build.} 
Participants will build a mechanism to trap the requests from generic
applications and pass them into BlurSense (a proof-of-concept for 
Android has already been built). 
Participants will build and manage an ``App Store'' for BlurSense to allow 
users to locate privacy filters they wish to apply.   These may range from 
sharing sensor data with researchers by reducing
the precision of sensor values, salting and hashing sensor values for 
anonymizing collected data, or completely denying access to individual 
(or all) sensors. For a particular sensor, a filter might perform an action such
as blurring the resolution of photos and video taken by the camera, 
removing access point information from WiFi scans, or
omitting the motion sensor data completely.   Security and privacy groups can
easily build and disseminate their own privacy filters they recommend to users
by adding them to BlurSense app store. Therefore, \blurname is able to handle 
the three categories of threats in Section~\ref{sec-threat}. 

\subsection{Design Framework}

\section{Conclusion}


% \section*{Acknowledgment}


\bibliographystyle{IEEEtran}
\bibliography{bibdata}

\end{document}