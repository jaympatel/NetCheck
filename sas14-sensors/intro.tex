\section{Introduction}
Smartphones and tablets are becoming the dominant way that people interact with
the physical world.  This trend is showing no sign of stopping.  Smartphones
overtook PC sales in 2011, and even tablets will outsell desktop PCs by
2013~\cite{phonesales}. As of October 2012, there were more than one billion
smartphones in use~\cite{smartphones-use}. By the end of 2013, the number of
mobile devices (like smartphones and tablets) is projected to surpass the number
of people on the planet~\cite{cisco-data}. Due to such ubiquity across
demographic and geographic spectrum, utilizing smartphones would substantially
benefit science and humanity.

\begin{comment}
First, smart devices produce immensely valuable data for researchers across
a wide variety of disciplines. As an interface between our everyday life and the
physical world, these devices present valuable data for 
critical scientific and social pursuits.
If properly harnessed, the embedded cellular, GPS, WiFi, Bluetooth, camera
sensors, etc.,
can objectively record information gathered from a user's
perspective for the benefit of all. For example, it is possible for researchers
to 
understand carrier and device performance 
differences~\cite{huang2010anatomizing}, increase wireless network performance
leveraging
user mobility~\cite{ravindranath2011improving}, improve navigation and
personalized 
trip management~\cite{thiagarajan2011accurate} and detect earthquakes and 
tsunamis accurately~\cite{faulkner2011next}. Estimation of traffic in an 
area can help transportation scientists refine usage models for environmental
impacts~\cite{lena2002elemental}. 
Health sciences, sociology, or any domain concerned with the movement of people
and the associated impact could be explored~\cite{goldman2009participatory}. 
But until now, there was no unified way to harvest these opportunities. 

Second, the computational power of modern smartphones is substantial.   
These devices provide significant
capacity and computing power at network edge --- 
a fact that has largely missed, except by criminals~\cite{botnet}.
Although other services like cloud computing provide substantial computing 
power, the location
of cloud infrastructure is less than ideal for many users because
of the latency when accessing many cloud datacenters.  Dedicated cloud 
infrastructure
also has limited host and network heterogeneity, limited scalability, and 
lack a deployment path to end-user systems. From the success of CDNs,
the trend has been toward pushing small datacenters close to end
users~\cite{edge-cloud}.
Smartphones, laptops, and tablets are already on the same network as
the user and thus minimize latency.
If researchers could use these devices to measure networks 
and serve content, this would provide deep insight into parts of the Internet 
that are currently invisible~\cite{bauermeasuring}. Therefore, harnessing the
power of smartphones 
provides unique opportunities for improving the quality of life
and enhancing the operation of modern society.

We propose to develop \sysname that will unleash the potential of
tablets and smartphones. 
Drawing on our experience (and user base) from our
Seattle platform, \sysname will control access to computational sandboxes
that runs unintrusively in the background of the tablets and smartphones of
users around the world.   
Any researcher can quickly and use the \sysname to easily access 
resources across a huge and diverse set of smartphones and tablets.   
(In fact thousands of researchers are already using our current 
testbed~\cite{SeattleClearinghouse}.)
This will provide unprecedented access to information about 
wireless access networks in diverse geographic regions, 
community roles and environmental impacts of individuals, 
and social patterns people  all around the world.
As a result, the \sysname will provide unparalleled research
capabilities for scientists and researchers across a wide variety of fields
\end{comment}

Smartphones and tablets in the modern age have powerful capabilities in
computing, communication, and sensing~\cite{cai2009defending}. Like desktop and
laptop machines, they can perform complex computing tasks according to user
commands. These smart devices can also communicate with each other through
wireless communications. Unlike regular computers, they have embedded
onboard sensors, such as accelerometer, gyroscope, GPS, and camera, which
enables the development of innovative mobile applications~\cite{sensor}.
Although the sensing capabilities enhance the convenience of user interfaces and
application usefulness, they also raise serious privacy
concerns~\cite{shabtai2010google}. For instance, through accessing sensor data,
malicious applications could retrieve sensitive information about the mobile
phone users, such as location, input passwords, and credit card numbers. They
even might be able to send these sensitive information to remote
attackers~\cite{xu2012taplogger, miluzzo2012tapprints, xu2009stealthy,
schlegel2011soundcomber, cai2011touchlogger, marquardt2011sp}. 

The current access control to the smartphone resources, such as sensor data, is
static and coarse-grained. We call this \textit{the first defensive line}. Take
the Android platform as an example, the access permissions are either granted or
denied completely during the installation of applications based on a request XML
manifest file. As a result, applications may ask for more permissions than
needed by the requirement. Once granted the requested permissions, applications
have access to those resources permanently. Another issue of the current work is
that they require modifications to the Android platform, such as
in~\cite{conti2011crepe, hornyack2011these}. This increases the cost of
maintenance and cannot be used in legacy systems. To protect smartphone
users' privacy, we propose a \textit{second defensive line} behind the first
line: a dynamic, fine-grained access control mechanism, which allows to
define and add privacy filters from user space~\cite{blur, Cappos_CCS_10}.


Specifically, we will implement a framework of reference monitors, to enforce
mandatory access control to sensor data in real time. The subjects of our
framework will be the processes of applications in user space. The objects will
be the sensor data on smart devices. If a process of any application needs to
access any of the sensor data, our second defensive line will come into play,
mediating every access to sensor data as follows. 

According to the semantics of the access requests and the current context,
different reference monitors will be triggered, calculating a sensitivity score
for each request from the applications, which has got through the first
defensive line. If the request is for highly sensitive sensor data, then the
request might be simply rejected. Otherwise, if the request is for medium-level
sensitive sensor data, the request might get through, but the returned data is
bogus. If the request is for low-sensitive sensor data, then the return results
need to be processed, e.g., by obfuscation, and the obfuscation ratio is
proportional to the sensitivity of the request: if the request is more
sensitive, then the returned data will be obfuscated in a more significant way.
If the request is for non-sensitive sensor data, then the unprocessed sensor
data will be returned.